\documentclass{resume}
\usepackage{zh_CN-Adobefonts_external, tabu, multirow, linespacing_fix, cite, titlesec, graphicx, amsmath, fontawesome5,eso-pic}

% Adjust section and subsection spacing
\titlespacing*{\section}{0pt}{0.5ex plus 0.2ex minus 0.2ex}{0.5ex plus 0.2ex}
\titlespacing*{\subsection}{0pt}{0.5ex plus 0.2ex minus 0.2ex}{0.5ex plus 0.2ex}

% Set paragraph and item spacing
\setlength{\parskip}{0.5ex}
\setlength{\itemsep}{0.5ex}

% 添加证件照背景图片
\AddToShipoutPictureBG{%
  \AtPageLowerLeft{%
    \put(456,701){%
      \includegraphics[width=1.15in]{avatar}
    }%
  }%
}

\begin{document}
\pagestyle{empty}
\name{李胤桥}
\basicInfo{
    \email{865562832@qq.com} \textperiodcentered\ 
    \phone{152-2510-8555} \textperiodcentered\ 
    \text{2001年4月生}
  }
\basicInfo{具备较好英语能力,计算机运开和人际沟通能力}

\section{\faGraduationCap\ 教育背景}
\datedsubsection{\textbf{华中师范大学,University of Wollongong (伍伦贡大学)QS162}}{2023年9月 -- 2025年6月}
\textbf{计算机技术,Computer Science(计算机科学) 双硕士学位} \\
主修科目:Python数据结构,项目管理,SQL,机器学习,密码学,数据挖掘,高阶网络安全 \\
荣获\textbf{2024-2025学年全额奖学金(8万)},雅思考试6.5\textbf{(阅读8听力7)} \\
担任研究生党支部书记,主持学部日常党务工作,获华中师范大学优秀研究生干部

\datedsubsection{\textbf{郑州航空工业管理学院}}{2019年9月 -- 2023年6月}
材料科学与工程——功能材料,工学学士(GPA3.44),英语(双学位) \\
\textbf{校级三好学生,校级二等奖学金,全国大学生英语竞赛二等奖}

\section{\faUsers\ 实习经历}
\datedsubsection{\textbf{上海磐松私募基金管理有限公司 运维工程师}}{2024年6月 -- 2024年9月}
\begin{itemize}
  \item 对象存储工作:针对生产环境的代码、日志、数据库等文件,利用共有云服务进行容灾;完成前期方案调研、价格调研、采购对接、部署落地、参数调优
  \item Python运维的代码优化:针对某py程序Crash后需要重新运行某一部分函数;利用Python的argparse模块以命令行的方式重新运行某段代码
  \item 服务器扩容工作:针对MongoDB数据库的慢查询问题进行排障,结论为缓存命中率低,进行Linux服务器内存扩容;针对Mysql数据库主从同步的固态硬盘空间需求,对Windows服务器进行RAID 5扩容
  \item 数据商数据部署工作:将数据商提供Docker-Compose部署方式,编写Dockerfile、利用k8s的Deployment部署方式部署。完成三个数据商、一个日内(T+0)交易软件的测试环境部署工作
\end{itemize}

\section{\faUsers\ 项目经历}
\datedsubsection{\textbf{基于多模态和rPPG技术的驾驶疲劳监测方法}}{2023年9月 -- 2025年6月}
计算机视觉,深度学习,Python,两项国家自然科学基金项目立项,一项结题
\begin{itemize}
  \item 基于视觉提取rPPG生理信号,获取心率,心率变异性,呼吸等相关数据
  \item 对卷积神经网络处理的多模态特征进行特征提取和处理,通过LSTM实现时序特征共享
  \item 通过特征级融合与决策及融合对多模态特征融合提取处理
  \item 3DCNN+时序归一化+SimAM注意力机制,以减半的参数实现与PhysNet模型相近的性能
  \item 项目成果《LightweightPhys: A Lightweight and Robust Network for Remote Photoplethysmography Signal Extraction》在Journal of Advanced Digital Communications(JADC)以一作身份发表
\end{itemize}

\datedsubsection{\textbf{机场跑道块状异物清除机器人}}{2020年4月 -- 2021年9月}
用于机场跑道的基于路径规划算法和计算机视觉的智能清扫机器人,基于OpenCV库和树莓派搭建
\begin{itemize}
  \item 作为Leader带领团队从零完成项目构建,推进和协调工作,多次担任路演角色,临场经验丰富
  \item 取得专利“ 一种机场跑道块状异物清除装置 ”立项,专利号:202120932618.7 
  \item 作为第一年立项的项目获郑州航院挑战杯一等奖,河南省挑战杯二等奖,河南省互联网+二等奖
\end{itemize}

\section{\faHeartO\ 科研经历}
\datedsubsection{\textbf{Automatic Knowledge Graph Construction over Efficient Information Extraction Networks\\构建在高效信息提取网络上的自动知识图谱, IEIR 2023会议 }}{2023年7月 -- 2023年11月}
\begin{itemize}
  \item 项目背景:为了预训练模型的训练成本,提高运行速度,在不降低命名实体识别准确率的情况下,用最低成本构建特定小样本数据集的知识图谱
  \item 解决方案:CNN+Bi-LSTM+Attention机制,实现了与RoBERTa+CRF模型几乎相同的0.8243的F1-score,但运行时间只有后者的一半
\end{itemize}

\end{document}